\documentclass[12pt,oneside]{article}
\usepackage{geometry}                                % See geometry.pdf to learn the layout options. There are lots.
\usepackage{listings}				% Perm ite utilizar lenguajes de programacion dentro de latex
\geometry{a4paper}                                           % ... or a4paper or a5paper or ... 
%\geometry{landscape}                                % Activate for for rotated page geometry
%\usepackage[parfill]{parskip}                    % Activate to begin paragraphs with an empty line rather than an indent
\usepackage{graphicx}                                % Use pdf, png, jpg, or epsß with pdflatex; use eps in DVI mode
                                                                % TeX will automatically convert eps --> pdf in pdflatex                
\usepackage{amssymb}

\usepackage[spanish]{babel}                        % Permite que partes automáticas del documento aparezcan en castellano.
\usepackage[utf8]{inputenc}                        % Permite escribir tildes y otros caracteres directamente en el .tex
\usepackage[T1]{fontenc}                                % Asegura que el documento resultante use caracteres de una fuente apropiada.

\usepackage{hyperref}                                % Permite poner urls y links dentro del documento
\usepackage{listings}
\title{Ejercicios de Programación - Sebesta}
\author{Lenguajes de Programación - ESPOL}

%\date{}                                                        % Activate to display a given date or no date

\begin{document}
\maketitle

\section{Introducción}
Las respuestas propuestas en este repositorio son producto del trabajo de los estudiantes de la materia ``Lenguajes de Programación'' de la ESPOL, correspondientes a las preguntas del libro de Robert Sebesta, Concepts of Programming Languages.

\section{Preguntas y Respuestas}

\subsection{Capítulo 5: Nombres, Enlaces y Alcances.}
\input{c5p4}
\subsection{Pregunta 5: Escriba una función en C que incluya las siguienes líneas de código: \newline
x = 21; \newline
int x; \newline
x = 42; \newline
Corra el programa y explique los resultados. Reescriba el mismo código en C++ y Java.}

Ninguno de los siguientes dos bloques de código compilan en C. En el listing 1, la variable 'x' aún no esta declarada al usarse en la primera asignación.
En el listing 2, hay una nueva declaración de la variable 'x', generando una contradicción.

\lstset{language=C}          % Set your language (you can change the language for each code-block optionally)

\begin{lstlisting}[caption= Pregunta 5 Capítulo 5, label=amb, frame=single]  % Start your code-block
  
#include<stdio.h>

main()
{
	x = 21;
	int x;
	x = 42;

	printf("%d", x);
	getch();

}
\end{lstlisting}
\begin{lstlisting}[caption= Pregunta 5 Capítulo 5, label=amb, frame=single]  % Start your code-block
  
#include<stdio.h>

main()
{
	int x;
	x = 21;
	int x;
	x = 42;

	printf("%d", x);
	getch();

}
\end{lstlisting}

Ninguno de los siguientes dos bloques de código compilan en C++. En el listing 3, la variable 'x' aún no esta declarada al usarse en la primera asignación.
En el listing 4, hay una nueva declaración de la variable 'x', generando una contradicción.

\lstset{language=C++}          % Set your language (you can change the language for each code-block optionally)

\begin{lstlisting}[caption= Pregunta 5 Capítulo 5, label=amb, frame=single]  % Start your code-block
  
#include <iostream>
using namespace std;

int main(void) {
	
	x = 21;
	int x;
	x = 42;
	cout << x << endl;
	system("pause");
	return 0;
}
\end{lstlisting}

\begin{lstlisting}[caption= Pregunta 5 Capítulo 5, label=amb, frame=single]  % Start your code-block
  
#include <iostream>
using namespace std;

int main(void) {
	int x;
	x = 21;
	int x;
	x = 42;
	cout << x << endl;
	system("pause");
	return 0;
}
\end{lstlisting}

Ninguno de los siguientes dos bloques de código compilan en Java. En el listing 5, la variable 'x' aún no esta declarada al usarse en la primera asignación.
En el listing 6, hay una nueva declaración de la variable 'x', generando una contradicción.

\lstset{language=Java}          % Set your language (you can change the language for each code-block optionally)

\begin{lstlisting}[caption= Pregunta 5 Capítulo 5, label=amb, frame=single]  % Start your code-block
  
public static void main(String[] args) {
        x = 21;
        int x;
        x = 42;
        
        System.out.println(x);
    }
\end{lstlisting}

\begin{lstlisting}[caption= Pregunta 5 Capítulo 5, label=amb, frame=single]  % Start your code-block
  
public static void main(String[] args) {
        int x;
        x = 21;
        int x;
        x = 42;
        
        System.out.println(x);
    }
\end{lstlisting}
\input{c5p6}
\input{c5p7}
\subsection{Capítulo 7: Expressions and Assignment Statements}
\input{c7p1}
\input{c7p2}
\begin{itemize}
\item {\bf Pregunta 3} \\\\
Escribe un programa en tu lenguaje favorito que determine y muestre la precedencia y asociatividad de sus operadores aritmeticos y booleanos.\\
Lenguaje Java\\
\begin{lstlisting}[frame=single]  % Start your code-block
using System;
public static void main(String[] args) \{
        
        float a,b,c,d,e,res1,res2, res3;
        res1=res2=res3=0;
        
        a=8;
        b=4;
        c=3;
        d=5;
        e=0;
        res1=a/b*c+d;
        res2=a*b/c+d;
        if(e!=0)\{
            res3=e*b*(c+d);
        \}
        System.out.println("El resultado 1 es: "+res1);
        System.out.println("El resultado 2 es: "+res2);
        System.out.println("El resultado 3 es: "+res3);
\}

\end{lstlisting}

Se sabe que en Java: La Exponenciación tiene precedencia 1(mayor), cambio de signo(-) e identidad(+) tienen precedencia 2, division y multiplicacion tienen precedencia 3, la suma y resta tienen precedencia 4 ,etc.\\
Los operadores también pueden tener la misma precedencia. En este caso, la asociatividad determina el orden en que deben actuar los operadores. \\
La asociatividad puede ser de izquierda a derecha o de derecha a izquierda.\\
En el resultado 1(res1), primero se resuelve los operadores de mayor precedencia. En este caso, tenemos la division y multiplicacion de igual precedencia.\\
Como ambas tienen la misma precedencia y asociatividad desde la izquierda, lo que significa que los operadores de la izquierda se procesan (division)antes que los operadores de la derecha(multiplicacion), quiere decir:\\
Primero, realiza a/b, luego ese resultado * c, y finalmente el resultado de la multiplición se le suma d. \\\\
En el resultado 2 (res2), como (/,*) tienen misma precedencia, primero se resuelve el operador de la izquierda (*), luego la división. Finalmente la suma de menor precedencia\\\\
En el resultado 3, se hace una evaluación de corto circuito. Cada vez que e sea diferente de 0, realiza la operación; caso contrario es 0\\\\\\

OUTPUT\\
El resultado 1 es: 11.0\\
El resultado 2 es: 15.666667\\
El resultado 3 es: 0.0\\

\item {\bf Pregunta 4} \\\\
Escribir un programa en Java que exponga la regla para el orden de evaluación de operando cuando uno de los operandos es un método call.\\
Lenguaje Java\\
\begin{lstlisting}[frame=single]  % Start your code-block
public class JavaApplication2 \{
    final static int num=5;
    static int a=5;
    /**
     * @param args the command line arguments
     */
    public static void main(String[] args) \{
        
          a = fun1()+a; 
          System.out.println(a); 
    \}

    static int fun1() \{
            a = 17;
            return 3;
       \}

\}

\end{lstlisting}
OUTPUT\\
20\\
Como podemos ver en el main en la 1era linea, fun1() retorna 3 y actualiza la variable global 'a'=17 y luego suma 17+3 asignando 20 a 'a'. Esto se debe a que en JAVA el operador '+' tiene asociatividad desde la izquierda.\\
En cambio si fuese:  a = a + fun1();. En este caso primero 'a'=5 + fun1() que retorne 3; dando una asignación de 8 a la variable a.

\item {\bf Pregunta 5} \\\\
Lenguaje C++
\begin{lstlisting}[frame=single]  % Start your code-block

#include<stdio.h>
#include<conio.h>
#include<stdlib.h>
#include<windows.h>
#include<math.h>

int fun1();

extern int a = 10;
void main()\{
	
	a = fun1()+a;  // a = a+fun1();
	printf("\%d", a);
	getch();
\}

int fun1() \{
	a = 17;
	return 3;
\}
\end{lstlisting}

OUTPUT\\
20\\
Como podemos ver en el main en la 1era linea, comparado con JAVA Y Si Shard en vez de que fun1() retorne 3 y actualize la variable global 'a' a 17,y luego sumarlos y asignarle 20 a 'a'. En C++, sea a = fun1()+a  ó  a = a+fun1(), llamado de función en la izq o derecha del operador en este caso '+'; siempre al llamar fun1(), este va actualizar la variable global a =17 y luego va sumarle 3, asignando un valor de 20 a  la variable 'a'. 


\item {\bf Pregunta 6} \\\\
Lenguaje Si Sharp\\
\begin{lstlisting}[frame=single]  % Start your code-block
class Program
    \{
        static int a = 5;
        static void Main(string[] args)
        \{
            a = a +fun1();
            Console.WriteLine(a);
            Console.ReadLine();
       \}

        static int fun1()
        \{
            a = 17;
            return 3;
        \}
    }
\end{lstlisting}
OUTPUT\\
8\\ 
Como podemos ver en Si Sharp también se cumple la regla de la asociativad. En este caso el operador '+' asoicativdad desde la izquierda. Primero a=5, luego le suma 3; cuyo resultado que es 8 se le asigna a la variable a.\\
En cambio si fuese:  a = fun1()+a. En ese caso fun1() retorna 3 y actualiza la variable global 'a'=17 y luego suma 17+3 asignando 20 a la varaible a.


\end{itemize}

\subsection{Capítulo 9: SubProgramas.}
\input{c9p4}
%\input{c5p6}

        




% Continuar con los siguientes capítulos y ejercicios:
% Ch6: 1, 2, 7
% Ch7: 1 - 6, 9
% Ch8: 3, 4, 5
% Ch9: 1, 5
% Recuerden que todos corresponden a las secciones de "Programming Exercises".

\end{document}
